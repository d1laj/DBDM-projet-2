\documentclass[french]{beamer}

\usepackage[utf8]{inputenc}
\usepackage[T1]{fontenc}
\usepackage{lmodern}
\usepackage{amsmath, amssymb}
\usepackage{graphicx}

\usepackage{babel}


%CHOIX DU THEME et/ou DE SA COULEUR
% => essayer différents thèmes (en décommantant une des trois lignes suivantes)
%\usetheme{PaloAlto}
%\usetheme{Madrid}
\usetheme{Copenhagen}

% => il est possible, pour un thème donné, de modifier seulement la couleur
%\usecolortheme{crane}
%\usecolortheme{seahorse}

%\useoutertheme[left]{sidebar}


%Pour le TITLEPAGE
\title{Projet de DBDM 2}
\subtitle{Football Data Challenge}
\author{Simonaitis et Lajou}
\date{\today}

\begin{document}

\begin{frame}
	\maketitle
\end{frame}

\begin{frame}
	\begin{center}
		Our dataset : Football Data challenge
		\includegraphics[scale=0.5]{LegaSerieAlogoTIM.png}
	\end{center}
\end{frame}

\begin{frame}
	Data from 1520 matches of serie A.\\
	The following seasons are given:\\
	\begin{tabbing}
	~~~~\= 2008-2009\\
	\>	2009-2010\\
	\>	2010-2011\\
	\>	2013-2014\\
	\end{tabbing}
	Our test set is composed of the following seasons:\\
	\begin{tabbing}
	~~~~\= 2014-2015\\
	\>	2015-2016\\
	\end{tabbing}
\end{frame}


\begin{frame}
	For each match we have:
	\begin{enumerate}
		\item An ID of the match
		\item The date of the match
		\item The home team and the away team
		\item The result for the matches in the train set (Home wins, Draw, Away wins)
		\item A set of odds
	\end{enumerate}
\end{frame}


\begin{frame}
	The odds for 6 differents sources.
	Each is composed of 3 numbers.\\
	Example: Sampdoria vs Juventus\\
	\begin{center}
		\begin{tabular}{lll}
			Home odds & Draw odds & Away odds \\
			8 & 4 & 1.45\\
			7.25 & 4 & 1.48\\
			5.5 & 4 & 1.55\\
			6 & 4 & 1.53\\ 
			7 & 4 & 1.5\\ 
			7.5 & 4 & 1.55
		\end{tabular}	
	\end{center}
\end{frame}


\begin{frame}
	Un environnement \texttt{frame} pour chaque \emph{diapositive}.
	\visible<2>{Chaque diapo pouvant contenir plusieurs \emph{couches}.}
\end{frame}





\begin{frame}
	Un environnement \texttt{frame} pour chaque \emph{diapositive}.
	\visible<2>{Chaque diapo pouvant contenir plusieurs \emph{couches}.}
\end{frame}


\begin{frame}{On peut mettre un titre : Sommaire}
	\tableofcontents
\end{frame}

\section{Section 1}
\begin{frame}{La section 1 commence}
	blabla
\end{frame}

\begin{frame}
	Un \textbf<2,3>{texte} en gras.
	\visible<3>{Un texte visible sur la 3\ieme{} couche}
\end{frame}

\begin{frame}{Titre (facultatif)}
\framesubtitle{Sous titre (facultatif aussi)}
	\begin{block}{Remarque}
	Un bloc
	\end{block}

	\begin{alertblock}{Proposition}
	Un bloc alerte
	\end{alertblock}

	\begin{exampleblock}<2>{Exemple}
	Un bloc exemple qui est visible sur la 2\ieme{} couche : $f(x)=2x$.
	\end{exampleblock}
\end{frame}

\section{Section 2}
\begin{frame}{La section 2 commence}
\begin{itemize}
	\item<1-> On peut cliquer sur les titres de la barre de gauche pour naviguer dans les sections du pdf (essayez !).
	\item<2-> On peut changer le ``look'' du beamer, en changeant de thème. Retournez dans le fichier source et compilez avec les autres thèmes proposés (il existe énormément de thèmes; seuls trois sont proposés dans le source).
\end{itemize}

\end{frame}

\end{document}
